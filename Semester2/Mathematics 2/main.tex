\documentclass{book}
\usepackage{graphicx}
\usepackage[english]{babel}
\usepackage{amsthm}
\usepackage{amsmath}
\usepackage{amssymb}
\usepackage{amsfonts}
\usepackage{physics}
\usepackage{tikz}
\usepackage[a4paper, margin=1in]{geometry}
\geometry{a4paper, margin=1in}
\usepackage{xcolor}
\graphicspath{ {./images/} }
\usepackage{svg}
\usepackage{bm}
\usepackage{appendix}
\renewcommand{\cleardoublepage}{\clearpage}
\tikzset{>=latex} % for LaTeX arrow head
\usepackage{xcolor}
{
\colorlet{veccol}{orange!90!black}
\colorlet{myblue}{blue!60!black}
\tikzstyle{vector}=[->,thick,veccol]
\def\R{1.4}
\def\r{0.03}
\def\N{9}
%\bm{\odot}%otimes
\def\RC{0.14}
\def\Cout{
  \tikz[baseline=-2.5]{ \draw[line width=0.7,myblue] (0,0) circle (\RC);
                        \fill[myblue] (0,0) circle (0.028); }
}
\def\Cin{
  \tikz[baseline=-2.5]{ \draw[line width=0.7,myblue] (0,0) circle (\RC)
                                                     (-135:0.75*\RC) -- (45:0.75*\RC)
                                                     (135:0.75*\RC) -- (-45:0.75*\RC); }
}
\def\null{\color{myblue}{0}}

}
\newtheorem*{theorem}{Theorem}
\title{Mathematics 2}
\author{Dominik Szablonski}

\begin{document}

\maketitle

\tableofcontents

\chapter{Partial Differentiation and Multiple Integration}
The differential function of a function $f = f(x,y)$ is given by,
\begin{equation}
    \dd{f} = \pdv{f}{x}\dd{x} + \pdv{f}{y}\dd{y}.
\end{equation}
\section{Taylor Series}
The Taylor series for a function $f(x_1,x_2,\ldots,x_m)$ about a point $(x_1^0,x_2^0,\ldots,x_m^0)$ is,
\begin{equation}
    f(x_1,x_2,\ldots,x_m) = f(x_1^0,x_2^0,\ldots,x_m^0) + \sum_{k=1}^{\infty}\frac{1}{k!}\left((x_1-x_1^0)\pdv[]{f}{x_1}+\cdots+(x_m + x_m^0)\pdv[]{f}{x_m}\right)^kf(x_1,x_2,\ldots,x_m),
\end{equation}
where the power on the bracket applies to the amount of times each differential is applied to $f$. Each partial derivative is evaluated about $(x_1^0,x_2^0,\ldots,x_m^0)$.
\section{Multiple Integration}
We are able to integrate over multiple variables. This allows for finding quantities such as a volume or area. 
\begin{equation}
    V = \int f(x,y,z)\dd{\tau} = \int \int \int f(x,y,z) \dd{x}\dd{y}\dd{z} = \int \left( \int \left( \int f(x,y,z) \dd{x}\right)\dd{y}\right)\dd{z}.
\end{equation}
\subsection{Areas and volumes in different coordinate systems}
\subsubsection{2D Plane Polar Coordinates}
\begin{align}
    x&=r\cos\theta & y&=r\sin\theta &\dd{A}&=r\dd{r}\dd{\theta}
\end{align}
\subsubsection{3D Cylindrical Polar Coordinates}
\begin{align}
    \text{Curved Surface: }\dd{A}&=r\dd{\theta}\dd{z} & \text{Top Surface: } \dd{A}&=r\dd{\theta}\dd{r} & \text{Volume: } \dd{V}=r\dd{r}\dd{\theta}\dd{z}
\end{align}
\subsubsection{3D Spherical Polar Coordinates}
\begin{align}
    \dd{A} &= r^2\sin\theta\dd{\theta}\dd{\phi} & \dd{V}&=r^2\sin\theta\dd{r}\dd{\theta}\dd{\phi}
\end{align}
There is an important identity required when working with polar co-ordinates,
\begin{equation}
    \sin\theta\dd{\theta} = -\dd{(\cos\theta)}.
\end{equation}
The limits for the volume of a sphere are,
\begin{equation}
    \int_0^{2\pi}\dd{\phi}\int_{0}^{\pi}\sin\theta\dd{\theta}\int_0^Rr^2\dd{r}.
\end{equation}
\subsection{Moment of Inertia}
The moment of inertia is given by,
\begin{equation}
    I = \int r_{\perp}^2\dd{m},
\end{equation}
which can be rewritten as,
\begin{equation}
    I = \rho\int r_{\perp}^2\dd{V},
\end{equation}
where $\rho$ is the density of the volume. For non-trivial axis, we use,
\begin{equation}
    r_{\perp} = |\vu{a}\cross\vb{p}|,
\end{equation}
where $\vu{a}$ is the vector in the direction of of the axis, and $\vb{p}$ is a generic vector such that,
\begin{equation}
    \vb{p} = r_i\vb{e}_i.
\end{equation}
\subsection{The Jacobian}
When converting between co-ordinate frames, we cannot always simply replace variables. We must use,
\begin{equation}
    \dd{x}\dd{y}\dd{z} = |J(u,v,w)|\dd{u}\dd{v}\dd{w}.
\end{equation}
Where, we can define,
\begin{align}
    x & = f(u,v,w) & y & = g(u,v,w) &z & = h(u,v,w) 
\end{align}
and then, the Jacobian is for a 2D system, the determinant of,
\begin{equation}
    \begin{vmatrix}
        \pdv{f}{u} & \pdv{f}{v} \\
        \pdv{g}{u} & \pdv{g}{v}
    \end{vmatrix}
\end{equation}
and for a 3D system,
\begin{equation}
    \begin{vmatrix}
         \pdv{f}{u} & \pdv{f}{v} & \pdv{f}{w} \\
        \pdv{g}{u} & \pdv{g}{v} & \pdv{g}{w} \\
        \pdv{h}{u} & \pdv{h}{v} & \pdv{h}{w}
    \end{vmatrix}.
\end{equation}
\chapter{Fields}
A field is an entity whose value depends on position. A field may be either \textbf{scalar} or \textbf{vector}. The direction of vector fields may also depend on position. \textbf{Scalar fields} are often represented by \textbf{contour lines}, which conenct the $x$ and $y$ points. \textbf{Vector fields} are often visualised using field lines. We can calculate the equations for these field lines by considering a vector field, $V = V_x \vu{i} + V_y\vu{j}$, and solving for,
\begin{equation}
    \dv{y}{x} = \frac{V_y}{V_x}.
\end{equation}
\section{Gradient $\grad$}
We can define the infinitesimal change in a scalar field as,
\begin{equation}
    d\psi = \grad \psi \cdot \dd{\vb{r}},
\end{equation}
where we define the \textbf{gradient} operator as,
\begin{equation}
    \grad = \pdv{}{r_i}\vb{e}_i.
\end{equation}
There are some properties of grad which we must be aware of. For a scalar field, $f(x,y,z)$,
\begin{itemize}
    \item $\grad f$ is a vector field.
    \item $\grad f$ represents the maximum rate of change of $f$.
    \item $\grad f$ is perpendicular to the contours of constant $f$.
    \item  The unit vector normal to a level surface is 
    \begin{equation}
        \frac{\grad f}{|\grad f|}.
    \end{equation}
\end{itemize}
There are two main types of questions where the grad operator is used,
\begin{enumerate}
    \item Finding the unit vector at $(x_0,y_0,z_0)$ which is normal to a level surface, $f(x,y,z)$.
    \begin{equation}
        \text{Obtain } \grad f(x,y,z) \to \vb{n} = \grad f(x_0,y_0,z_0) \to \vu{n} = \frac{\grad f(x_0,y_0,z_0)}{|\grad f(x_0,y_0,z_0)|}
    \end{equation}
    \item Find the rate of increase at $f(x_0,y_0,z_0)$ in the direction between, $(x_1,y_1,z_1)$ and $(x_2,y_2,z_2)$. 
    \begin{equation}
        \text{Eval. } \grad f(x_0,y_0,z_0) \to \dd{\vb{s}} = (\Delta x, \Delta y, \Delta z) \to \vu{u} = \frac{\dd{\vb{s}}}{|\dd{\vb{s}}|} \to \dv{f}{s} = \grad f(x_0,y_0,z_0) \cdot \vu{u}.
    \end{equation}
\end{enumerate}
where $\phi$ and $\psi$ are scalar fields and $K$ is a position independent constant scalar.
\subsection{Grad in polar coordinates}
\begin{align}
    \text{Cylindrical:} && \grad \psi & = \pdv{\psi}{r}\vu{r} + \frac{1}{r}\pdv{\psi}{\theta}\vu*{\theta} + \pdv{\psi}{z}\vu{z}, \\
    \text{Spherical:} && \grad \psi & = \pdv{\psi}{r}\vu{r} + \frac{1}{r}\pdv{\psi}{\theta}\vu*{\theta} + \frac{1}{r\sin \theta}\pdv{\psi}{\phi}\vu*{\psi}.
\end{align}
\section{Lagrangian Multipliers}
These are used when wanting to find the minimum or maximum of a field when a field $f$ is constrained by some function $g$. If $g$ is a constant constraint, then we can say that,
\begin{equation}
    \dd{g} = 0
\end{equation}
and for a minimum or maximum of $f$, similarly,
\begin{equation}
    \dd{f} = 0.
\end{equation}
We recall that the change in a scalar field along an elemental path is,
\begin{equation}
    \dd{\psi} = \grad \psi \cdot \dd{\vb{s}}.
\end{equation}
Thus, $\grad f$ and $\grad g$ are both perpendicular to $\dd{\vb{s}}$. Thus, $\grad f$ and $\grad g$ are parallel or anti parallel to each other. We can then state the relation,
\begin{equation}
    \grad f = \lambda \grad g
\end{equation}
which brings about a system of equations,
\begin{equation}
    \begin{split}
        \pdv{f}{x} - \lambda \pdv{g}{x} & = 0 \\
        \pdv{f}{y} - \lambda \pdv{g}{y} & = 0
    \end{split}
\end{equation}
We will want to include the function of $g$ when finding the solutions to this equation. 
\begin{enumerate}
    \item Find $x$,$y$, and $z$ in terms of $\lambda$.
    \item Substitute these into the expression for $g$.
    \item Substitute the value found for $g$ back into the expressions for $x$,$y$, and $z$ to find the coordinates of the maximum and minimum rate of change.
\end{enumerate}

\section{Div}
\begin{align}
    \text{Div}(\vb{F}) \equiv \div{\vb{F}}
\end{align}
The divergence describes the flux through an area of a  vector field. A field with 0 divergence is known as a \textit{solenoidal field} The total flux of a vector field through a volume composed of $i$ surfaces is given by the \textbf{divergence theorem},
\begin{equation}
\begin{split}
    \sum_i \int_{S_i}\vb{F}\cdot\dd{\vb{S}}_i = \int_V(\div{\vb{F}})\dd{V} \\
    \div{\vb{F}} = \frac{1}{|\vb{v}|}\int_S \vb{F}\cdot\dd{\vb{S}}
\end{split}
\end{equation}
Whether the flux is $+$ive, $-$ive or 0 will give us different information.\\\\
When $|\vb{F}| \equiv \text{const}$, we have,
\begin{equation}
    \div{\vb{F}} = 0.
\end{equation}
For \textit{increasing} $|\vb{F}|$, the total flux is given by,
\begin{equation}
    \int (F_R - F_L)\dd{s} 
\end{equation}
\begin{equation}
    F_R > F_L \therefore \div{\vb{F}} > 0 \implies \text{Net flow \textit{out of the volume.}}
\end{equation}
The field is known as a \textit{source} of flux.\\\\
For \textit{decreasing} $|\vb{F}|$,
\begin{equation}
    F_L > F_R \therefore \div{\vb{F}} < 0 \implies \text{Net flow \textit{into the volume.}}
\end{equation}
The field is then known as a sink of flux.
\\\\
For non-trivial fields, the values of Div still apply. 
\subsection{Div in polar coordinates}
For cylindrical polar coordinates,
\begin{equation}
    \div{\vb{A}} = \frac{1}{r}\pdv{}{r} \left(rA_r\right)+ \frac{1}{r}\pdv{}{\theta}A_{\theta} + \pdv{}{z}\left(A_z\right).
\end{equation}
For spherical polar coordiantes,
\begin{equation}
    \div{\vb{A}} = \frac{1}{r^2}\pdv{}{r}\left(r^2A_r\right) + \frac{1}{r\sin\theta}\pdv{}{\theta} \left(A_{\theta}\sin\theta\right) + \frac{1}{r\sin\theta}\pdv{}{\phi} \left(A_{\phi}\right)
\end{equation}
\subsection{Gauss' Law}
\begin{equation}
\begin{split}
    &\int\vb{E}\cdot\dd{\vb{S}} = \int_V(\div{\vb{E}})\dd{V} = \frac{1}{\epsilon_0}\int\dd{Q} \\
    &\div{\vb{E}} = \frac{1}{\epsilon_0}\dv{Q}{V} = \frac{\rho}{\epsilon_0}\\
\end{split}
\end{equation}
\section{Curl}
\begin{equation}
    \text{Curl}(\vb{F}) \equiv \curl{\vb{F}}
\end{equation}
We can interpret the physical nature of curl if we imagine dropped a ball into a field. If $(\curl{\vb{F}}) \neq 0$, then the ball will experience a torque. A field with 0 curl is known as an \textit{irrotational field}.
\subsubsection{Curl in polar coordinates}
For cylindrical polar coordinates,
\begin{equation}
    \curl{\vb{A}} = \frac{1}{r}\begin{vmatrix}
        \vu{r} & r\vu*{\theta} & \vu{k} \\
        \pdv{}{r} & \pdv{}{\theta} & \pdv{}{z} \\
        A_r & rA_{\theta} & A_z
    \end{vmatrix}
\end{equation}
For spherical polar coordinates,
\begin{equation}
    \curl{\vb{A}} = \frac{1}{r\sin\theta}\begin{vmatrix}
        \vu{r} & r\vu*{\theta} & r\sin\theta\vu*{\phi} \\
        \pdv{}{r} & \pdv{}{\theta} & \pdv{}{\phi} \\
        A_r & rA_{\theta} & r\sin\theta A_{\phi}
    \end{vmatrix}
\end{equation}

\section{The Laplacian Operator}
This is defined,
\begin{equation}
    \nabla^2 = \grad \cdot \grad.
\end{equation}
Such that,
\begin{equation}
    \nabla^2 = \sum_i \pdv[2]{}{r_i}.
\end{equation}
For a scalar field $\psi$, the Laplacian is,
\begin{equation}
    \nabla^2\psi = \sum_i\pdv[2]{\psi}{r_i}.
\end{equation}
For a vector field, $\vb{F}$, the Laplacian is given by,
\begin{equation}
    \nabla^2\vb{F} = \nabla^2A_i\vb{e}_i.
\end{equation}
\subsection{Laplacian in Polar Coordinates}
In cylindrical polar coordinates,
\begin{equation}
    \nabla^2\psi = \frac{1}{r}\pdv{}{r}\left(r\pdv{\psi}{r}\right) + \frac{1}{r^2}\pdv[2]{\psi}{\theta} + \pdv[2]{\psi}{z}.
\end{equation}
In spherical polar coordinates,
\begin{equation}
    \nabla^2\psi = \frac{1}{r^2}\pdv{}{r}\left(r^2\pdv{\psi}{r}\right) + \frac{1}{r^2\sin\theta}\pdv{}{\theta}\left(\sin\theta\pdv{\psi}{\theta}\right)+\frac{1}{r^2\sin\theta}\pdv[2]{\psi}{\phi}
\end{equation}

\section{Maxwell's Equations}
\begin{align}
    & \div{\vb{E}} = \frac{\rho}{\epsilon_0} & \text{Gauss' Law}&\\
    & \div{\vb{B}} = 0 & \text{No magnetic monopoles}&\\
    & \curl{\vb{E}} = -\pdv{\vb{B}}{t} & \text{Faraday's Law}&\\
    &\curl{\vb{B}} = \mu_0\vb{J} + \mu_0\epsilon_0\pdv{\vb{E}}{t} & \text{Ampere's Law}&
\end{align}
The wave equation for electromagnetic waves can then be derived,
\begin{equation}
    \nabla^2\vb{E} = \frac{1}{c^2}\pdv[2]{\vb{E}}{t}.
\end{equation}
\section{Helmholtz Decomposition}
If you have a smooth, rapidly varying field that vanishes faster than $\frac{1}{r}$ as $r \to \infty$,
\begin{equation}
    \vb{v} = \curl{\vb{\vb{A}}} + \grad\phi.
\end{equation}
We know that,
\begin{align}
    &\div{(\curl{\vb{A}})} = 0 & \curl{\grad\phi} = 0&\\
    &\text{"B-mode"} & \text{"E-Modes"}&
\end{align}

\section{Surface Integrals}
When we integrate over a surface, we usually integrate over a vector, given as,
\begin{equation}
    \dd{\vb{S}} = \vu{n}\dd{S}
\end{equation}
where $\vu{n}$ is a unit vector which is perpendicular and away from the surface. If our surface is described by a function $f$, we can compute $\vu{n}$ as,
\begin{equation}
    \vu{n} = \frac{\grad f}{\left|\grad f\right|}.
\end{equation}
The elemental area $\dd{S}$ can be rewritten in Cartesian coordinates if we consider that the elemental surface is inclined at some angle, $\theta$ to the $x-y$ plane. We then say,
\begin{equation}
    \begin{split}
        \dd{x}\dd{y} & = \dd{S}\cos\theta = \dd{S}(\vu{n}\cdot\vu{k}) \\
        & = \dd{\vb{S}}\cdot\vu{k}.
    \end{split}
\end{equation}
\subsection{Solid Angles}
We often wish to integrate over a solid angle. The infinitesimal element of a solid angle is given by,
\begin{equation}
    \dd{\Omega} = \sin\theta\dd{\theta}\dd{\phi}.
\end{equation}
Surface integrals over solid angles have the infinitesimal area element,
\begin{equation}
    \dd{\Omega} = \frac{\vu{r}\cdot\dd{\vb{S}}}{r^2}.
\end{equation}
\subsection{Flux}
The flux through of a vector field $\vb{A}$ through a surface $S$ is given by,
\begin{equation}
    \text{FLUX} = \int_S\vb{A}\cdot\dd{\vb{S}}.
\end{equation}
Often, it is a lot easier to do this using the Divergence theorem covered in the next section.
\section{Divergence Theorem}
\begin{theorem}
    The total flux of a vector $\vb{A}$ through a closed surface $S$ is related to the divergence of a vector field inside a volume $V$ by,
    \begin{equation}
        \int_S \vb{A}\cdot\dd{\vb{S}} = \int_V\div{\vb{A}}\dd{V}.
    \end{equation}
\end{theorem}\noindent 
This theorem can also be applied to scalar fields,
\begin{equation}
    \int_S \psi \dd{\vb{S}} = \int_V\grad\psi \dd{V}.
\end{equation}
\subsection{Applications of Divergence Theorem}
\subsubsection{Maxwell's First Equation}
Gauss' law can be given as follows,
\begin{equation}
	\int_S \vb{E}\cdot\dd{\vb{S}} = \frac{1}{\epsilon_0}\int_V \rho \dd{V}.
\end{equation}
Re-writing this in terms of the divergence theorem,
\begin{equation}
	\begin{split}
		\int_V \div{\vb{E}}\dd{V} &= \frame{1}{\epsilon_0}\int_V \rho \dd{V}\\
		\div{\vb{E}} & = \frac{\rho}{\epsilon_0},
	\end{split}
\end{equation}
which is Maxwell's first equationn of electromagnetism.
\subsubsection{The Continuity Equation}
This is a conservation equation, given by,
\begin{equation}
	\dv{\rho}{t} + \div{\vb{J}} = 0
\end{equation}
where $\rho$ is an amount of some quantity, $q$, per unit volume, and $\vb{J}$ is the flux per unit time of $q$.
\section{Line Integrals}
For a curve $C$ defined by a function $f$ and a small length element $\dd{l}$, the scalar line integral is defined by,
\begin{equation}
	\int_C f\dd{l}.
\end{equation}
For 2D line integrals, we can define,
\begin{equation}
	(\dd{l})^2 = (\dd{x})^2 + (\dd{y})^2
\end{equation}
and,
\begin{equation}
	l = \int_C \dd{l} = \int_{x_0}^{x_1} \sqrt{1 + \left(\dv{y}{x}\right)^2}\dd{x}.
\end{equation}
For a curve defined by parameters, i.e., $x = g(t)$, $y = h(t)$, we write,
\begin{equation}
	l = \int_{t_0}^{t_1}\sqrt{\left(\dv{x}{t}\right)^2 + \left(\dv{y}{t}\right)^2}\dd{x}.
\end{equation}
In polar co-ordinates,
\begin{equation}
	\dd{l} = \sqrt{r^2 + \left(\dv{r}{\theta}\right)^2}\dd{\theta}.
\end{equation}
\subsection{Vector line integrals}
For a vector $\vb{A}$, we define the samll length element vector as,
\begin{align}
	\dd{\vb{l}} = \dd{x}\vu{i} \dd{u}\vu{j} && \text{Cartesian} \\
	\dd{\vb{l}} = \dd{r}\vu{r} + r\dd{\theta}\vu*{\theta} && \text{Polar}.
\end{align}
The line integral is then given by,
\begin{equation}
	l = \int_C \vb{A} \cdot \dd{\vb{l}}.
\end{equation}
\subsection{Circulation of Vector Fields}
If we wish to find the circulation of a vector $\vb{A}$ along a certain path, $c$, we should compute $\vb{A}\cdot\dd{\vb{l}}$. If we define $c = c(x,y,z)$, then we should substitute appropriately so that we are able to integrate over $\vb{A}\cdot\dd{\vb{l}}$.
\\\\
To find the circulation over polar coordinates, we want to first parametrise the curve, such that,
\begin{align}
	x = g(\theta) && y = f(\theta)
\end{align}
and obtain a vector, $\vb{c}$,
\begin{equation}
	\vb{c} = (g(\theta),f(\theta)).
\end{equation}
We can then rewrite $\vb{A}$ in terms of the components of $\vb{c}$ and use,
\begin{equation}
	\vb{A}\cdot\dd{\vb{l}} = \vb{A}(\vb{c}) \cdot \dv{\vb{c}}{\theta}\dd{\theta}
\end{equation}
and perform the integral. For clockwise circulation this is $0 \to 2\pi$ and for anti-clockwise this is $2\pi \to 0$.
\subsection{Stoke's Theorem}
The circuation of a vector fied $\vb{B}$ is related to the surface integral of the curl by,
\begin{equation}
	\oint_C \vb{B}\cdot\dd{\vb{l}} = \int\int_S (\curl{\vb{B}})\cdot\dd{\vb{S}}
\end{equation}
We can use the right hand rule to determine the direction of $\dd{\vb{S}}$.
\subsection{Conditions of Stoke's Theorem}
\begin{enumerate}
	\item Curve is continuous,
	\item The surface is simply connected,
	\item There are no non-integrateable singularities.
\end{enumerate}
\section{Conservative Forces}
Conservative forces satisfy the condition that $\curl{\vb{F}} = 0$. Thus,
\begin{enumerate}
	\item Work done is independent of path taken
	\item They have an associated scalar potential
	\item The work done along the path is equal to the potential between the two points
\end{enumerate}
We can find the force from a potential $V$ by,
\begin{equation}
	\vb{F} = -\grad{V}.
\end{equation}
We can reverse this to find the potential,
\begin{equation}
	V = - \int F_x \dd{x} = - \int F_y\dd{y} = - \int{F_z}\dd{z}
\end{equation}
we must remember to include a functions of $x$, $y$, and/or $z$ as our constant of integration, as well as $c$. We must then find the functions which are common to all integrals and combine them into a final expression for $V$.
\chapter{Fourier series}
A Fourier series can be used to describe a \textit{period function}, $P(t)$. In order for a function to be described by a Fourier series it must satisfy \textbf{Dirilecht conditions}:
\begin{enumerate}
	\item $P(t)$ must be bounded i.e., have finite values over the whole period of the function.
	\item $P(t)$ must only have a finite number of discontinuities and min/maxima over the period.
	\item $P(t)$ must be integrable over theh whole period of the function.
\end{enumerate}
Let us then state, even functions are those where,
\begin{equation}
	P(-t) = P(t)
\end{equation}
and can be described by sums of cosines, while odd functions are those where,
\begin{equation}
	P(-t) = -P(t)
\end{equation}
and can be described purely by sums of sines. The Fourier series works because the sine and cosine functions are orothogonal, meaning,
\begin{equation}
	\int_0^Tf(t)g(t) = 0.
\end{equation}
Let us further define,
\begin{align}
	T && \text{Period} \\
	f = \frac{1}{T} && \text{Frequency} \\
	\omega = 2\pi f = \frac{2\pi}{T} && \text{Angular frequency}.
\end{align}
\section{Trigonometric summation}
\begin{equation}
	P(t) = \frac{a_0}{2} + \sum_{n=1}^{\infty}a_n\cos n\omega t + b_n \sin n \omega t.
\end{equation}
We define the three coeficients as such,
\begin{align}
	a_0 & = \frac{\omega}{\pi}\int_{-\frac{\pi}{\omega}}^{\frac{\pi}{\omega}}P(t)\dd{t} \\
	a_n & = \frac{\omega}{2\pi}\int_{-\frac{\pi}{\omega}}^{\frac{\pi}{\omega}}P(t)\cos n \omega t\dd{t}\\
	b_n & = \frac{\omega}{2\pi}\int_{-\frac{\pi}{\omega}}^{\frac{\pi}{\omega}}\sin n\omega t \dd{t}.
\end{align}
The average value of the periodic function over the period is given by $\frac{a_0}{2}$.
\section{Exponential Summation}
We can say exponential functions are orthogonal if,
\begin{equation}
	\int_0^T e^{-in\omega t}e^{in\omega t}\dd{t} = 0.
\end{equation}
We can further express the Fourier series of a function over the sum of the exponentials,
\begin{align}
	P(t) & = \sum_{n=-\infty}^{\infty}C_e^{in\omega t} n \in \mathbb{Z} \\
	C_n & = \frac{1}{T} \int_0^TP(t)e^{-in\omega t}\dd{t} \label{C}
\end{align}
The exponential form gives identical expressions to the trigonometric form. we can show this by splitting the summation,
\begin{equation}
	P(t) = C_0 + \sum_{n=1}^{\infty}\left[C_n(+n)e^{in\omega t} + C_n(-n)e^{-in\omega t}\right]
\end{equation} 
where $C_n(+n)$ is the value of the coefficaint evaluated at $+n$ and vice versa. 
\\\\
For the case that the function is even, then $C_n$ is purely real, $C_n(+n) = C_n(-n)$, and the summation becomes,
\begin{equation}
	2\sum{n=1}^{\infty}C_n(+n)\cos{n\omega t}.
\end{equation}
If the function is odd, $C_n$ is purely imaginary, $C_n(+n) = -C_n(-n)$, and,
\begin{equation}
	2i\sum_1^{\infty}C_n(+n)\sin n\omega t.
\end{equation}
\appendix
 \chapter{Vector Calculus Identities}
\section{Grad Identities}
\begin{align}
    & \grad(\psi + \phi) = \grad\psi + \grad\phi \\
    & \grad(K\psi) = K\grad\psi \\
    & \grad(\psi\phi) = \psi\grad\phi + \phi\grad\psi 
\end{align}
\section{Curl Identities}
\begin{align}
    & \curl{(\vb{A}+\vb{B})} = \curl{\vb{A}} + \curl{\vb{B}} \\
    & \curl{(\phi\vb{A})} = \phi(\curl{\vb{A}})+(\grad\phi)\cross\vb{A} \\
    & \curl{(\vb{A}\cross\vb{B})} = \vb{A}(\div{\vb{B}})-\vb{B}(\div{\vb{A}})+(\vb{B}\cdot\grad)\vb{A}-(\vb{A}\cdot\grad)\vb{B}
\end{align}
\section{Div Identities}
\begin{align}
    & \div{(\vb{A}+\vb{B})} = \div{\vb{A}} + \div{\vb{B}} \\
    & \div{(\phi\vb{A})} = (\grad\phi)\cdot\vb{A} +\phi(\div{\vb{A}}) \\
    & \div{(\vb{A}\cross\vb{B})} = \vb{B}\cdot(\curl{\vb{A}})-\vb{A}\cdot(\curl{\vb{B}})
\end{align}
\section{Combination Identities}
\begin{align}
    & \grad \cross \grad\phi = 0 \\
    & \grad \cdot (\curl{\vb{A}}) = 0 \\
    & \curl{(\curl{\vb{A}})} = \grad(\div{\vb{A}}) - \nabla^2\vb{A}.
\end{align}
\chapter{Vector Calculus Proofs}
\section{$\div{(g\vb{F})} = (\grad g)\cdot \vb{F} + g(\div{\vb{F}})$}
\begin{proof}
    \begin{equation*}
        \begin{split}
            \div{(g\vb{F})} & = (gF_1)_x + (gF_2)_y + (gF_3)_z \\
            & = g_xF_1 + g_y F_2 + g_z F_3 + g\left[(F_1)_x + (F_2)_y + (F_3)_z\right] \\
            & = \begin{pmatrix}
                g_x \\ g_y \\ g_z
            \end{pmatrix} \cdot \begin{pmatrix}
                F_1 \\ F_2 \\ F_3
            \end{pmatrix} + g\begin{pmatrix}
                (F_1)_x \\ (F_2)_y \\ (F_3)_z 
            \end{pmatrix}\\
            & = (\grad g)\cdot \vb{F} + g(\div{\vb{F}})
        \end{split}
    \end{equation*}
\end{proof}

\end{document}
