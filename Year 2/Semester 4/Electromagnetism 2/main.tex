\documentclass{book}
\usepackage{graphicx}
\usepackage[english]{babel}
\usepackage{amsthm}
\usepackage{amssymb}
\usepackage{amsfonts}
\usepackage{mdframed}
\usepackage{physics}
\usepackage{tikz}
\usepackage[a4paper, margin=1in]{geometry}
\geometry{a4paper, margin=1in}
\usepackage{xcolor}
\usetikzlibrary{arrows.meta}
\usetikzlibrary{angles,quotes}
\graphicspath{ {./images/} }
\usepackage{svg}
\usepackage{subcaption}
\usepackage{bm}
\usepackage{empheq}
\usepackage{cancel}
\usetikzlibrary{decorations.text}
\usepackage[most]{tcolorbox}
\usepackage{tensor}
%3D
\usepackage{mathtools}
\usepackage{booktabs}
\usepackage{array}
\newcolumntype{C}{>{$}c<{$}}
\usepackage{tikz-3dplot}
\usepackage{appendix}
\usepackage{pgfplots}
\usetikzlibrary{shapes.geometric}
\usetikzlibrary{calc,patterns,angles,quotes}
%Tikz Library
\usetikzlibrary{angles, quotes, intersections}
\usepackage[bb=dsserif]{mathalpha}
\usetikzlibrary{decorations.pathmorphing}

\tikzset{snake it/.style={decorate, decoration=snake}}

\usepackage{etoolbox} % ifthen
\usepackage[outline]{contour} % glow around text
\usetikzlibrary{calc} % for adding up coordinates
\usetikzlibrary{decorations.markings,decorations.pathmorphing}
\usetikzlibrary{angles,quotes} % for pic (angle labels)
\usetikzlibrary{arrows.meta} % for arrow size
\usepackage{xfp} % higher precision (16 digits?)

\usepackage{tcolorbox}

%https://osl.ugr.es/CTAN/macros/latex/contrib/tcolorbox/tcolorbox.pdf
\tcbuselibrary{breakable}
\tcbset{%any default parameters
	width=0.7\textwidth,
	halign=justify,
	center,
	breakable,
	colback=white    
}

\newenvironment{aside}
{\begin{mdframed}[style=0,%
		leftline=false,rightline=false,leftmargin=2em,rightmargin=2em,%
		innerleftmargin=0pt,innerrightmargin=0pt,linewidth=0.75pt,%
		skipabove=7pt,skipbelow=7pt]\small}
	{\end{mdframed}}

\renewcommand{\cleardoublepage}{\clearpage}

\title{Electromagnetism 2}
\author{Dominik Szablonski}
\newtheorem{law}{Law}
\newtheorem{klaw}{Law}
\newtheorem*{definition}{Definition}
\newtheorem*{theorem}{Theorem}

\pgfplotsset{compat=1.18}
\begin{document}
\maketitle

\tableofcontents

\chapter{Sources of Radiation}
Radiation is the phenomenon of energy being transported to an observer. Radiation must be disconnected from its source. Plane electromagnetic waves satisfy conditions of being radiation, and are thus known as free fields. The source of electromagnetic radiation must be charges as those generate the electric and magnetic fields. However, electromagnetic radiation is only generated through accelerating charges. For a free field, there must be locations where radiation propagates while $\rho = 0$.
\section{Charges}
Let us look more closely at why stationary and uniformly moving charges do not produce radiation.
\subsection{Stationary Charges}
For a stationary charge, we can write,
\begin{align}
	\vb{E} & = \frac{1}{4\pi\varepsilon_0}\frac{q(\vb{r} - \vb{r}')}{|\vb{r} - \vb{r}'|^3} \propto \frac{1}{|\vb{r} - \vb{r}'|^2} \\
	\vb{B} & = 0
\end{align}
We can clearly see that a stationary charge cannot radiate as it has no component in $\vb{b}$. Furthermore, the energy flow varies as $\vb{1}{r^2}$, which require $\vb{E}\propto\frac{1}{r}$ and $\vb{B}\propto\frac{1}{r}$ which neither field satisfies.
\subsection{Charges with Uniform Velocity}
We can perform a Lorentz boost in the case of a moving charge,
\begin{equation}
	\vb{E}(\vb{r}) = \frac{q}{4\pi \varepsilon_0}\frac{1-\beta^2}{(1 - \beta^2\sin^2\theta)}\frac{\vu{r}}{|\vb{r}|^2}.
\end{equation}
We find that the electric flux still varies as $\frac{1}{r^2}$, thus cannot . Furthermore, we can move into the charge's rest frame where it still varies as $\frac{1}{r^2}$. This is the case as charge is both a conserved and Lorentz invariant quantity.
\\\\
We could also make the an argument using the Biot-Savarte law,
\begin{equation}
	\vb{B}(\vb{r}) = \frac{\mu_0}{4\pi}\frac{q\vb{v}\cross\vu{r}}{\left|\vb{r}\right|^2}
\end{equation}
which still varies as $\frac{1}{r^2}$. Thus, we require acceleration of the charge for radiation to occur.
\section{Retarded Potentials}
If we recall eqs. \eqref{eq:A} and \eqref{eq:phi} of the sourced potentials, we can analyse more closely the static and dynamic cases to gain more insight into radiation phenomena.
\subsection{Statics}
In the static case, we clearly see that we recover Poisson's equation, which we can solve trivially.
\subsection{Time-Dependent case}
In the time dependent case, we require that the electric field propagates at the speed of light $c$. Thus, we wish to obtain a scalar potential $\phi(\vb{r},t)$ sue to a volume of charge $\rho\dd{V}'$ at $\vb{r}'$. We must use the charge density which existed in that volume at an earlier time $\tau$,
\begin{equation}
	\tau = t - \frac{|\vb{r} - \vb{r}'|}{c}
\end{equation}
where $\frac{|\vb{r} - \vb{r}'|}{c}$ is the time taken for information ot travel from $\vb{r}'$ to $\vb{r}$. We can call $\tau$ the \textit{retarded time}. We can thus write down the retarded potentials,
\begin{align}
	&\phi(\vb{r},t) = \frac{1}{4\pi\varepsilon_0}\int_V \frac{\rho(\vb{r}', \tau)}{|\vb{r} - \vb{r}'|}\dd{V}' \\
	&\vb{A}(\vb{r},t) = \frac{\mu_0}{4\pi}\int_V\frac{\vb{J}(\vb{r}',\tau)}{|\vb{r}- \vb{r}'|}\dd{V}'
\end{align}
\appendix
\chapter{Revision Equations}
\section{Potentials}
\begin{tcolorbox}[colback=red!5!white,colframe=red!75!black,title=Static Electric Potential]
	\begin{align}
		\vb{E} &= - \grad \phi \\
		\laplacian{\phi} &= - \frac{\rho}{\varepsilon_0}
	\end{align}
\end{tcolorbox}
\begin{tcolorbox}[colback=blue!5!white,colframe=blue!75!black,title=Static Magnetic Potential]
	\begin{align}
		\vb{B} &= \curl{\vb{A}} \\
		-\curl{\vb{B}} &= \laplacian{\vb{A}} = - \mu_0 \vb{J}
	\end{align} 
	if we choose,
	\begin{align}
		\vb{A} \to \vb{A} + \grad{\psi} && \div{\vb{A}} = 0.
	\end{align}
\end{tcolorbox}
\begin{tcolorbox}[colback=green!5!white,colframe=green!75!black,title=Dynamic Potentials]
	\begin{align}
		\laplacian\vb{A} - \mu_0\varepsilon_0 \pdv{\vb{A}}{t} &= - \mu_0 \vb{J} \label{eq:A} \\
		\laplacian\vb{\phi} - \mu_0\varepsilon_0\pdv{\phi}{t} &= -\frac{\rho}{\varepsilon_0} \label{eq:phi}
	\end{align}
\end{tcolorbox}
\section{Gauges}
\begin{tcolorbox}[colback=red!5!white,colframe=red!75!black,title=Coloumb Gauge]
	\begin{equation}
		\div{\vb{A}} = 0
	\end{equation}
\end{tcolorbox}
\begin{tcolorbox}[colback=blue!5!white,colframe=blue!75!black,title=Lorenz Gauge]
	\begin{equation}
		\div{\vb{A}} + \frac{1}{c^2}\pdv{\phi}{t} = 0
	\end{equation}
\end{tcolorbox}
\section{Vector Identities}
\begin{tcolorbox}[colback=blue!5!white,colframe=blue!75!black,title=Laplacian of a Vector]
	\begin{equation}
		\curl{\curl{\vb{A}}} = \grad\left(\div{\vb{A}}\right) - \laplacian{\vb{A}}
	\end{equation}
\end{tcolorbox}
\end{document}
