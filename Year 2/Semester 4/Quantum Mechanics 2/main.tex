\documentclass{book}
\usepackage{graphicx}
\usepackage[english]{babel}
\usepackage{amsthm}
\usepackage{amssymb}
\usepackage{amsfonts}
\usepackage{mdframed}
\usepackage{physics}
\usepackage{tikz}
\usepackage[a4paper, margin=1in]{geometry}
\geometry{a4paper, margin=1in}
\usepackage{xcolor}
\usetikzlibrary{arrows.meta}
\usetikzlibrary{angles,quotes}
\graphicspath{ {./images/} }
\usepackage{svg}
\usepackage{subcaption}
\usepackage{bm}
\usepackage{empheq}
\usepackage{cancel}
\usetikzlibrary{decorations.text}
\usepackage[most]{tcolorbox}
\usepackage{tensor}
%3D
\usepackage{mathtools}
\usepackage{booktabs}
\usepackage{array}
\newcolumntype{C}{>{$}c<{$}}
\usepackage{tikz-3dplot}
\usepackage{appendix}
\usepackage{pgfplots}
\usetikzlibrary{shapes.geometric}
\usetikzlibrary{calc,patterns,angles,quotes}
%Tikz Library
\usetikzlibrary{angles, quotes, intersections}
\usepackage[bb=dsserif]{mathalpha}
\usetikzlibrary{decorations.pathmorphing}

\tikzset{snake it/.style={decorate, decoration=snake}}

\usepackage{etoolbox} % ifthen
\usepackage[outline]{contour} % glow around text
\usetikzlibrary{calc} % for adding up coordinates
\usetikzlibrary{decorations.markings,decorations.pathmorphing}
\usetikzlibrary{angles,quotes} % for pic (angle labels)
\usetikzlibrary{arrows.meta} % for arrow size
\usepackage{xfp} % higher precision (16 digits?)

\usepackage{tcolorbox}

%https://osl.ugr.es/CTAN/macros/latex/contrib/tcolorbox/tcolorbox.pdf
\tcbuselibrary{breakable}
\tcbset{%any default parameters
	width=0.7\textwidth,
	halign=justify,
	center,
	breakable,
	colback=white    
}

\newenvironment{aside}
{\begin{mdframed}[style=0,%
		leftline=false,rightline=false,leftmargin=2em,rightmargin=2em,%
		innerleftmargin=0pt,innerrightmargin=0pt,linewidth=0.75pt,%
		skipabove=7pt,skipbelow=7pt]\small}
	{\end{mdframed}}

\renewcommand{\cleardoublepage}{\clearpage}

\title{Quantum Mechanics 2}
\author{Dominik Szablonski}
\newtheorem{law}{Law}
\newtheorem{klaw}{Law}

\newtcbtheorem{Definitions}{Definition}%
{colback=blue!5!white,colframe=blue!75!black,width=\textwidth,fonttitle=\bfseries}{}

\newtcbtheorem{Theorems}{Theorem}%
{colback=red!5!white,colframe=red!75!black,width=\textwidth,fonttitle=\bfseries}{}

\newtheorem*{theorem}{Theorem}


\setlength\parindent{0pt}
\pgfplotsset{compat=1.18}
\begin{document}
\maketitle

\tableofcontents

\chapter{Orbital Angular Momentum}
\section{Basics of QM}
Let us recall some basic facts of quantum mechanics.
\\\\
The expectation value of an observable $\mathcal{A}$ with an associated operator $\hat{A}$ is given by,
\begin{equation}
	\left<\hat{A}\right> = \bra{\Psi}\hat{A}\ket{\Psi} = \int\Psi^*\hat{A}\Psi\dd{\vb{r}}.
\end{equation}
The fundamental position, momentum, and angular momentum operators are defined as follows,
\begin{Definitions}{Fundamental Operators}{}
	\begin{align}
		\hat{\vb{r}} &= x\vu{x} + y\vu{y} + z\vu{z} \\
		\hat{p} &= -i\hbar \grad \\
		\hat{L}_i &= \varepsilon_{ijk}\hat{r}_j\hat{p}_k
	\end{align}
\end{Definitions}
The Hamiltonian is defined,
\begin{Definitions}{Hamiltonian}{}
	\begin{equation}
		\hat{H} = \hat{T} + \hat{V} = -\frac{\hbar^2}{2m}\laplacian + V(\vb{r}, t).
	\end{equation}
\end{Definitions}
We obtain the wavefunction $\Psi$ by solving the TDSE,
\begin{Definitions}{Time Dependent Schrodinger Equation}{}
	\begin{equation}
		i\hbar\pdv{\Psi(\vb{r},t)}{t} = \hat{H}\Psi(\vb{r},t).
	\end{equation}
\end{Definitions}
For the static case, this reduces to the TISE,
\begin{equation}
	\hat{H}\Psi = E \Psi. \label{eq:TISE}
\end{equation}
If $\Psi(\vb{r},0)$ is written in the energy eigenbasis, i.e., $\Psi(\vb{r},0) = \sum_i c_i\ket{E_i}$, then the time-dependent solution is trivial,
\begin{equation}
	\Psi(\vb{r},t) = \sum_i c_i\ket{E_i} \exp\left(\frac{-iE_it}{\hbar}\right).
\end{equation}
\subsection{The Simple Harmonic Oscillator}
The SHO has a Hamiltonian,
\begin{equation}
	\hat{H} = - \frac{\hbar^2}{2m}\dv[2]{x} + \frac{1}{2}m\omega^2x^2
\end{equation}
with energy eigenvalues,
\begin{equation}
	E_n = \left(n + \frac{1}{2}\right)\hbar \omega \label{eq:1D SHO E}
\end{equation}
and has normalised Eigenfunctions,
\begin{equation}
	\psi_n(x) = \left(\frac{1}{n!2^na\sqrt{\pi}}\right)H_n\left(\frac{x}{a}\right)\exp\left(-\frac{x^2}{2a^2}\right)
\end{equation}
where $a = \sqrt{\hbar/m\omega}$ and $H_n(x/a)$ is a Hermite polynomial.
\subsection{Simple Perturbation Theory}
In simple perturbation theory, we write the Hamiltonian as,
\begin{equation}
	\hat{H} = \hat{H}_0 + \hat{V}
\end{equation}
where the Hamiltonian $\hat{H}_0$ is trivial and for which we already have obtained its eigenfunction $\psi$ and eigenvalues $E_n^{(0)}$. We then use this to find the expectation value of the total Hamiltonian,
\begin{equation}
	\left<\hat{H}\right> = \bra{\psi}\hat{H}_0 + \hat{V}\ket{\psi} = E_n^{(0)} + \Delta E.
\end{equation}
Writing this more explicitly,
\begin{Definitions}{First Order Perturbation Theory}{}
	\begin{equation}
		E_n = E_n^{(0)} + \bra{\psi}\hat{V}\ket{\psi}
	\end{equation}
\end{Definitions}
\section{Particle in 2D SHO}
The Hamiltonian of the 2D SHO is given by,
\begin{equation}
	\hat{H}\psi(x,y) = -\frac{\hbar^2}{2m}\left(\pdv[2]{x} + \pdv[2]{y}\right) + \frac{1}{2}m\omega (x^2 + y^2)\psi(x,y) = E\psi(x,y)
\end{equation}
We can separate this Hamiltonian into its $x$ and $y$ components,
\begin{align}
	\hat{H}_x = -\frac{\hbar^2}{2m}\pdv[2]{x} + \frac{1}{2}m\omega x^2 && \hat{H}_y -\frac{\hbar^2}{2m}\pdv[2]{y} + \frac{1}{2}m\omega y^2.
\end{align}
We know the solution to the 1D SHO, as by eq. \eqref{eq:1D SHO E}. We can intuit that the total solution of the 2D Hamiltonian will be a product of the two 1D wavefunctions. This comes from the fact that to add probabilities, we multiply the probability densities. So, we write,
\begin{equation}
	\begin{split}
		\hat{H}\psi_{n_x}(x)\psi_{n_y}(y) & = \left(\hat{H}_x + \hat{H}_y\right)\psi_{n_x}(x)\psi_{n_y}(y) \\
		& = \left(\hat{H}_x\psi_{n_x}(x)\right) \psi_{n_y}(y) + \psi_{n_x}(x)\left(\hat{H}_y\psi_{n_y}(y)\right) \\
		& = \left(n_x + \frac{1}{2}\right)\hbar\omega \psi_{n_y}(y) + \left(n_y + \frac{1}{2}\right)\hbar \omega \psi_{n_x}(x) \\
		& = \left(n_x + n_y + 1\right)\hbar \omega \psi_{n_x}(x)\psi_{n_y}(y) \\
		\implies E_{n_x,n_y} &= (n_x + n_y + 1)\hbar \omega.
	\end{split}
\end{equation}
\subsection{Degeneracy}
This is when there is more than one state with the same energy. The degeneracy $D$ is the number of energy states that share the same energy. Non-degenerate states are those with $D = 1$. 
\section{Orbital Angular Momentum}
The angular momentum in given direction in a classical system is given by,
\begin{equation}
	L_i = \varepsilon_{ijk}r_jp_k.
\end{equation}
The angular momentum operator in quantum mechanics is thus,
\begin{equation}
	\hat{L}_i = \varepsilon_{ijk}\hat{r}_j\hat{p}_k.
\end{equation}
We are particularly interested in the case where $i=z$, in which case the operator becomes,
\begin{equation}
	\hat{L}_z = \hat{x}\hat{p}_y - \hat{y}\hat{p}_x = -i\hbar\left(x\pdv{y} - y\pdv{x}\right).
\end{equation}
Let us consider this operator in plane polar coordinates, $(r, \theta)$. We have,
\begin{align}
	x = r\cos\theta && y = r\sin\theta
\end{align}
Let us consider the following,
\begin{equation}
	\begin{split}
	\pdv{\theta} & = \pdv{x}{\theta}\pdv{x} + \pdv{y}{\theta} \pdv{y} = -r\sin\theta \pdv{x} + r\cos\theta\pdv{y}\\
	& = -y\pdv{x} + x\pdv{y}.
\end{split}
\end{equation}
So, in plane polars,
\begin{Definitions}{Angular Momentum Operator in Z}{}
	\begin{equation}
	\hat{L}_z = -i\hbar \pdv{\theta}.
\end{equation}
\end{Definitions}
\subsection{Eigenfunctions and Eigenvalues of $\hat{L}_z$}
We wish to consider the following,
\begin{equation}
	\hat{L}_z \Theta(\theta) = L_z\Theta(\theta).
\end{equation}
So,
\begin{equation}
	-i\hbar\dv{\Theta}{\theta} = L_z \Theta \label{eq:24}
\end{equation}
which we can solve trivially,
\begin{equation}
	\Theta(\theta) = Ae^{\frac{L_z\theta}{\hbar}} \label{eq:khf}
\end{equation}
where $A = \frac{1}{\sqrt{2\pi}}$ is a normalisation constant. We require a cyclic boundary condition, such that $\Theta(\theta) = \Theta(\theta + 2\pi)$. So,
\begin{equation}
	\begin{split}
	Ae^{\frac{iL_z(\theta + 2\pi)}{\hbar}} & = A e^{\frac{iL_z\theta}{\hbar}} \\
	e^{\frac{iL_z2\pi}{\hbar}} & = 1.
\end{split} \label{eq:hehe}
\end{equation}
Not all values of $L_z$ satisfy the eq. \eqref{eq:hehe}, so we have to impose the following restriction,
\begin{equation}
	L_z = \hbar m, \hspace{1em} m \in \mathbb{Z}
\end{equation}
and thus, we can write the angular momentum eigenfunction as,
\begin{Definitions}{Angular Momentum Eigenfunction}{}
	\begin{equation}
		\Theta_m(\theta) = \frac{1}{\sqrt{2\pi}} e^{im\theta}
	\end{equation}
\end{Definitions}

\subsection{Angular Momentum of the 2D SHO}
We wish to express eigenfunctions of the 2D SHO as eigenfunctions of angular momentum. we will find that we require a combination of all degenerate eigenfunctions for a givevn $D$ in order to represent angular momentum eigenfunction. Observing the ground state,
\begin{equation}
	\Psi_{00}(x,y) = e^{-\frac{x^2}{2a^2}}\cdot e^{-\frac{y^2}{2a^2}} = e^{-\frac{r^2}{2a^2}}, \hspace{2em} a^2 = \frac{\hbar}{2m}.
\end{equation}
Applying the angular momentum operator we find,
\begin{equation}
	\hat{L}_z\Psi_{00} = 0 \cdot \Psi_{00}
\end{equation}
which holds, as $0$ is an allowed value of $m$. The first excited states of $D=2$ are given by,
\begin{align}
	\Psi_{10} = x e^{-\frac{x^2}{2a^2}} \cdot e^{-\frac{y^2}{2a^2}} && \Psi_{01} = e^{-\frac{x^2}{2a^2}} \cdot ye^{-\frac{y^2}{2a^2}}
\end{align}
which we combine to form,
\begin{equation}
	\begin{split}
	\Psi_{\pm} &= \Psi_{10} \pm i\Psi_{01} \\
	& = \left[r\cos\theta \pm ir\sin\theta\right]e^{-\frac{r^2}{2a^2}} = re^{\pm i\theta}e^{-\frac{r^2}{2a^2}}. \label{eq:eigenfunction}
	\end{split}
\end{equation}
Applying $\hat{L}_z$ to eq. \eqref{eq:eigenfunction},
\begin{equation}
	\hat{L}_z\Psi_{\pm} = \pm \hbar \Psi_{\pm}
\end{equation}
$\implies$ $\Psi_{\pm}$ is an eigenfunction of $\hat{L}_z$ with eigenvalues $\pm \hbar$. Furthermore, $\Psi_{\pm}$ is an eigenfunction of $\hat{H}$, so $\hat{H}$ and $\hat{L}_z$ commute. This allows for the 2D SHO to be described by \textit{good quantum numbers}. These satisfy the following,
\begin{enumerate}
	\item Can be known simultaneously,
	\item Fully and uniquely specify the state of a system.
\end{enumerate}
For the 2D SHO, its good quantum numbers are $(n,m)$, where $n = n_x + n_y$. $n$ specifies the energy of the system (as by $E_n = (n + 1)\hbar \omega$), and $m$ specifies the angular momentum of the system (as by $L_z = m\hbar$). 
\end{document}
