\documentclass{book}
\usepackage{graphicx}
\usepackage[english]{babel}
\usepackage{amsthm}
\usepackage{amssymb}
\usepackage{amsfonts}
\usepackage[T1]{fontenc}
\usepackage[siunitx, RPvoltages]{circuitikz}
\usepackage{physics}
\usepackage{tikz}
\usepackage[a4paper, margin=1in]{geometry}
\geometry{a4paper, margin=1in}
\usepackage{xcolor}
\usetikzlibrary{arrows.meta}
\usetikzlibrary{angles,quotes}
\graphicspath{ {./images/} }
\usepackage{svg}
\usepackage{subcaption}
\usepackage{bm}
\usepackage{empheq}
\usetikzlibrary{decorations.text}
\usepackage[most]{tcolorbox}
\usepackage{tensor}
%3D
\usepackage{mathtools}
\usepackage{booktabs}
\usepackage{array}
\newcolumntype{C}{>{$}c<{$}}
\usepackage{tikz-3dplot}
\usepackage{appendix}
\usepackage{pgfplots}
\usetikzlibrary{shapes.geometric}
\usetikzlibrary{calc,patterns,angles,quotes}
%Tikz Library
\usetikzlibrary{angles, quotes, intersections}
\usepackage[bb=dsserif]{mathalpha}
\usetikzlibrary{decorations.pathmorphing}

\tikzset{snake it/.style={decorate, decoration=snake}}

\usepackage{etoolbox} % ifthen
\usepackage[outline]{contour} % glow around text
\usetikzlibrary{calc} % for adding up coordinates
\usetikzlibrary{decorations.markings,decorations.pathmorphing}
\usetikzlibrary{angles,quotes} % for pic (angle labels)
\usetikzlibrary{arrows.meta} % for arrow size
\usepackage{xfp} % higher precision (16 digits?)

\renewcommand{\cleardoublepage}{\clearpage}

\title{Lagrangian Dynamics}
\author{Dominik Szablonski}
\newtheorem{law}{Law}
\newtheorem{klaw}{Law}
\newtheorem*{definition}{Definition}
\newtheorem*{theorem}{Theorem}

\pgfplotsset{compat=1.18}
\begin{document}
\maketitle

\tableofcontents

\chapter{Lagrangian Mechanics}
\section{Introduction}
\begin{figure}
	\centering
	\begin{tikzpicture}
		\begin{axis}[axis lines=left, xlabel=$x$, ylabel=$V(x)$,]
			\addplot [
			domain=-10:10, 
			samples=100, 
			color=black,
			]
			{-0.33*x^4 + 58*x^3 + 2*x - 1};
		\end{axis}
	\end{tikzpicture}
	\caption{Arbitrary potential} \label{fig:arbitrary potential}
\end{figure}
We have up until now solved dynamical problems using Newton's laws. However, they are not applicable to all scenarios. The laws are the following,
\begin{enumerate}
	\item A body acted on with no forces moves with uniform velocity.
	\begin{itemize}
		\item Einstein's general relativity is a counterexample to this law.
	\end{itemize}
	\item The rate of change of momentum of a body is give by the total force acting on that body.
	\begin{itemize}
		\item In quantum mechanics, momentum is probabilistic.
	\end{itemize}
	\item Every action force has an equal and opposite reaction force.
	\begin{itemize}
		\item This is not true in collective phenomenon, e.g. confiment of quarks in a proton.
	\end{itemize}
\end{enumerate}
Lagrangian/Hamiltonian mechanics is able to solve all systems where Newtonian mechanics is applicable\footnote{Except for cases with friction, which will be omitted in this course.}, as well as alleviating Newtonian mechanics of its shortcomings.
\section{Review of classical mechanics and mathematicl preliminaries}
Consider a particle of mass $m$ moving in 1 dimension in an arbitrary, explicitly time-independent potential $V(x)$. The particle will feel a force $F$ due to the potential such that,
\begin{equation}
	F = -\pdv{V}{x}
\end{equation}
and has an acceleration such that,
\begin{equation}
	F = m\Ddot{x}.
\end{equation}
The kinetic energy of the particle is given by,
\begin{equation}
	T = \frac{1}{2}m\Dot{x}^2
\end{equation}
and the total energy,
\begin{equation}
	E = T + V
\end{equation}
is conserved. The momentum,
\begin{equation}
	p = m\Dot{x}
\end{equation}
is not conserved.
\subsection{Potential confusion}
We have stated that the potential is \textit{explicitly} time independent, however, we may have reason to suspect this is not the case. There are two ways to looks at this,
\begin{enumerate}
	\item \textbf{NO} $\to$ The potential does not depend on $t$ as we have not specified it to be so. Mathematically, we can express this via the partial derivative,
	\begin{equation}
		\pdv{V}{t} = 0.
	\end{equation}
	\item \textbf{YES} $\to$ The potential does depend on time, as the particle may move to a different position, and thus be in an area of different potential, i.e. the time dependence is \textit{implicit}. Mathematically, this is expressed by considering the total derivative,
	\begin{equation}
		\begin{split}
			\dv{V}{t} &= \pdv{V}{t} + \pdv{V}{x}\dv{x}{t} + \pdv{V}{\Dot{x}}\dv{\Dot{x}}{t}\\
			& = \pdv{V}{x}\dv{x}{t} =\pdv{V}{x}\Dot{x}.
		\end{split}
	\end{equation}
\end{enumerate}
We can extend this concept further to kinetic energy $T$ and total energy $E$, whose total derivatives with time are,
\begin{align}
	\dv{T}{t} & = m\Dot{x}\Ddot{x} \\
	\dv{E}{t} & = \dv{t}\left(T + V\right) = \Dot{x}\left(\pdv{V}{x} + m\Ddot{x}\right). \label{eq:total energy time derivative}
\end{align}
We can get some very interesting physics from eq. \eqref{eq:total energy time derivative}, and where \textit{Lagrange's approach} to mechanics falls in. Lagrange's method says to assume two solutions to eq. \eqref{eq:total energy time derivative},
\begin{align}
	\Dot{x} = 0 && \pdv{V}{x} + m\Ddot{x} = 0
\end{align}
where the latter is what we wish to investigate.
\section{Lagrangian and Lagranges Equation}
Let us define the \textit{Lagrangian},
\begin{equation}
	\boxed{L = T - V}.
\end{equation}
In the 1 dimensional case,
\begin{equation}
	\begin{split}
		L(x, \Dot{x}) & = \frac{1}{2}m\Dot{x}^2 - V(x) \\
		\pdv{L}{x} & = \underbrace{-\pdv{V}{x} = m\Ddot{x}}_{\substack{\text{Newton's seconod}\\\text{law}}} = \dv{}{t}\left(m\Dot{x}\right) = \dv{}{t}\left(\pdv{L}{\Dot{x}}\right).
		\end{split} \label{eq:lang derv 1}
\end{equation}
From eq. \eqref{eq:lang derv 1}, we obtain \textit{Lagrange's equation},
\begin{equation}
	\boxed{\dv{t}\left(\pdv{L}{\Dot{x}}\right) = \pdv{L}{x}}.
\end{equation}
\subsection{Generalised approach to solving a dynamical system}
\begin{enumerate}
	\item Work out the degrees of freedom of a system, that is,
	\begin{align}
		\left\{q_i\right\} && \text{Set of generalised coordinates} \\
		\left\{\Dot{q}_i\right\} && \text{Set of generalised velocities}.
	\end{align}
	\item Write down the Lagrangian,
	\begin{equation}
		L\left(\left\{q_i\right\},\left\{\Dot{q}_i\right\},t\right) = T - V.
	\end{equation}
	\item Derive the equations of motion using Lagrange's equation,
	\begin{equation}
		\dv{t}\underbrace{\left(\pdv{L}{\Dot{q}_i}\right)}_{\substack{\text{Generalised}\\\text{Momentum}}} = \overbrace{\pdv{L}{q_i}}^{\substack{\text{Generalised}\\ \text{Force}}} \label{eq: generalised lagrangian}
	\end{equation}
\end{enumerate}
\subsection{Generalised Coordinates}
Before we proceed to talking about generalised coordinates, lets find the Lagrangian in plane polar coordinates. We require the following velocities to compute the Lagrangian,
\begin{align}
	v_r = \Dot{r} && v_{\theta} = r\Dot{\theta} && v^2 = \Dot{r}^2 + r^2\Dot{\theta}^2.
\end{align}
The kinetic energy is then,
\begin{equation}
	T = \frac{1}{2}m(\Dot{r}^2 + r^2\Dot{\theta}^2),
\end{equation}
and the Lagrangian,
\begin{equation}
	L = \frac{1}{2}m(\Dot{r}^2 + r^2\Dot{\theta}^2) - V(r, \theta).
\end{equation}
Since there are two coordinates in the Lagrangian, we have 2 equations of motion, given by,
\begin{align}
	\dv{t}\left(\pdv{L}{\Dot{r}}\right) &= \pdv{L}{r}, \\
	\dv{t}\left(\pdv{L}{\Dot{\theta}}\right) &= \pdv{L}{\theta}.
\end{align}
We find that each coordinate (degree of freedom) is treated democratically. If we were to do this in spherical polar coordiantes, we would find that we would get 3 equations of motion, etc. Instead of being restricted to a set coordinate system, let us describe the lagrangian for a \textit{generalised coordinate system}. Below are the generalised quantities we should be aware of:
\begin{align}
	\text{Generalised coordinates} && q_i(t) \\
	\text{Generalised velocities} && \Dot{q}_i(t) \\
	\text{Generalised momenta} && p_i(t) = \pdv{L}{\Dot{q}_i}
\end{align}
for which the generalised Lagrangian equation is given by eq. \eqref{eq: generalised lagrangian}.
\chapter{The Calculus of Variations}
\section{The Principle of Least Action}
We define the action $S$,
\begin{equation}
	\boxed{S = \int L\dd{t}}
\end{equation}
and we assert that a system evolved over time such that the action is minimised. This is a fundamental law of physics. We have that $S = S\left[x(t)\right]$, indicating it is a function of a function.
\\\\
Our general problem is that we wish to find a function $x(t)$ which minimises,
\begin{equation}
	S\left[x(t)\right] = \int_{t_1}^{t_2} L\left(x(t),\Dot{x}(t),t\right)\dd{t}
\end{equation}
subject to the boundary conditions,
\begin{align}
	x(t_1) &= x_1 \\
	x(t_2) & = x_2
\end{align}
for single values functions of time.\\\\
\begin{figure}
	\centering
	\begin{tikzpicture}[>=latex]
		\begin{axis}[
			axis x line = center,
			axis y line = center,
			xtick = {0, 1, 2, 3, 4, 5},
			ytick = {0, 1, 2, 3, 4, 5},
			ticks = none,
			xmin=0,
			ymin=0,
			xmax=5.5,
			ymax=5.5,
			xlabel={$t$},
			ylabel={$x(t)$}
			]
			\addplot [black, mark = *, nodes near coords={$x_1$},every node near coord/.style={anchor=east}] coordinates {(1, 1)};
			\addplot [black, mark = *, nodes near coords={$x_1$},every node near coord/.style={anchor=east}] coordinates {(4, 4)};s
		\end{axis}
	\end{tikzpicture}
\end{figure}
Let us suppose a function $x(t)$ minimises $S$. Let us now consider a small displacement $\delta x(t)$ on $x(t)$, such that we have a modified path $x(t) + \delta x(t)$. This modified path must obey the boundary conditions, so
\begin{align}
	\delta x(t_1) &= 0\\
	\delta x(t_2) & = 0.
\end{align}
We must find an $x(t)$ such that $\delta x (t)$ does not change the value of $S$. We can write the modified Lagrangian,
\begin{equation}
	L\left(x(t) + \delta x(t), \Dot{x}(t) + \dv{t}\delta x(t), t\right) = L + \delta L.
\end{equation}
We can then calculate the modified action,
\begin{equation}
	\delta S = \int_{t_1}^{t_2} \delta L \dd{t} = \int_{t_1}^{t_2}\left[\pdv{L}{x}\delta x(t) + \pdv{L}{\Dot{x}}\dv{t}\left(\delta x(t)\right)\right]\dd{t}. \label{eq:modified path}
\end{equation}
We can integrate the second term of eq. \eqref{eq:modified path} by parts,
\begin{equation}
	\begin{split}
	\int_{t_1}^{t_2}\pdv{L}{\Dot{x}} \dv{t}\left(\delta x(t)\right)\dd{t}& = \underbrace{\left[\pdv{L}{x}\delta x(t)\right]_{t_1}^{t_2}}_{0 \text{ by BCs}} -\int_{t_1}^{t_2} \delta x(t) \dv{t}\left(\pdv{L}{\Dot{x}}\right)\dd{t} \\
	& = -\int_{t_1}^{t_2}\dv{t}\left(\pdv{L}{\Dot{x}}\right)\delta x(t)\dd{t} \dd{t} \\
	\implies \delta S & = \int_{t_1}^{t_2}\left[\pdv{L}{x} - \dv{t}\left(\pdv{L}{\Dot{x}}\right)\right]\delta x(t) \dd{t} = 0.
	\end{split} \label{eq:blah balh}
\end{equation}
We can interpret eq. \eqref{eq:blah balh} in 2 ways,
\begin{enumerate}
	\item $\delta x(t)$ is carefully chosen such that the integrand is non-zero, but integrates to 0. However, since $\delta x(t)$ is arbitrary, we can choose it to not be so;
	\item The integrand is 0 everywhere.
\end{enumerate}
Given the second conclusion, we have that,
\begin{align}
	\delta x(t) = 0 && \boxed{\pdv{L}{x} - \dv{t}\left(\pdv{L}{\Dot{x}}\right) = 0} \label{eq:euler lagrange}
\end{align}
and we have thus derived the Euler-Lagrange equations.
\\\\
A furhter conclusion is that the principle of least action directly implies that $x(t)$ obeys Lagrange's equation, which directly implies Newton's laws.
\subsubsection{Principle of Least Action - More Generally}
Any function defined,
\begin{equation}
	f = f(y_i(x), y'_i(x), x)
\end{equation}
with an integral,
\begin{equation}
	S = \int_{x_1}^{x_2}f\dd{x}
\end{equation}
is minimised by,
\begin{equation}
	\dv{x}\left(\pdv{f}{y_i'}\right) = \pdv{f}{y_i}. \label{eq:sols}
\end{equation}
\chapter{Hamiltonian Mechanics}
We are motivated to reformulate our approach to mechanics to directly include the conservation of momentum. However, because of the nature of partial derivatives, simply rewriting the Lagrangian to be $L = L(q, p, t)$ violates the principle of least action. We require a new quantity, which we will call the Hamiltonian, which we write $H = H(\left\{q_i\right\},\left\{p_i\right\}, t)$ which we can obtain by the \textit{Legendre Transform}.
\\\\
Let us begin by writing,
\begin{equation}
	\dd{L} = \pdv{L}{q}\dd{q} + \pdv{L}{\dot{q}}\dd{\Dot{q}} + \pdv{L}{t}\dd{t}. \label{eq:inf lagrangian}
\end{equation}
We want to find a function $H(q, p, t)$ where,
\begin{equation}
	p \equiv \pdv{L}{\dot{q}}
\end{equation}
where,
\begin{equation}
	\dd{H} = \pdv{H}{q}\dd{q} \pdv{H}{p} + \pdv{H}{t}\dd{t} . \label{eq: akjh}
\end{equation}
We have that the Legendre transformation of $L$ is $H$, so,
\begin{equation}
	H = p\dot{q} - L. \label{eq:hamiltonian}
\end{equation}
From eq. \eqref{eq:hamiltonian},
\begin{equation}
	\dd{H} = p\dd{\dot{q}} + \dot{q}\dd{p} - \dd{L}.
\end{equation}
and by eq. \eqref{eq:inf lagrangian},
\begin{equation}
	\begin{split}
	\dd{H} = p\dd{\dot{q}} + \Dot{q}\dd{p} - \pdv{L}{q} \dd{q} - \underbrace{\pdv{L}{\Dot{q}}\dd{\Dot{q}}}_{p\dd{\dot{q}}} - \pdv{L}{t}\dd{t} \\
	\Dot{q}\dd{p} - \pdv{L}{q}\dd{q} - \pdv{L}{t}\dd{t}. \label{eq:inf hamilt}
	\end{split}
\end{equation}
By comparing the coefficients of eqs. \eqref{eq: akjh} and \eqref{eq:inf hamilt}, we obtain \textit{Hamilton's equations},
\begin{align}
	\pdv{H}{p} = \Dot{q} && \pdv{H}{q} = -\dot{p} && \pdv{H}{t} = -\pdv{L}{t}. \\
\end{align}
Most generally, for a system of $N$ degrees of freedom $\left\{q_i\right\}$, and Lagrangian, $L(\left\{q_i\right\}, \left\{\Dot{q}_i\right\}, t)$, the hamiltonian is defined,
\begin{equation}
	\boxed{H(\left\{q_i\right\}, \left\{p_i\right\}, t) = \sum_ip_i\Dot{q_i} - L(\left\{q_i\right\}, \left\{\Dot{q}_i\right\}, t)}
\end{equation}
for,
\begin{equation}
	p_i = \pdv{L}{\Dot{q}_i},
\end{equation}
and Hamilton's equations,
\begin{align}
	\boxed{\pdv{H}{p_i} = \Dot{q}_i} \\
	\boxed{\pdv{H}{q_i} = \Dot{p}_i} \\
	\boxed{\pdv{H}{t} = - \pdv{L}{t}},
\end{align}
which produce $2N$ 1$^{\text{st}}$ order differential equations, in contrast to the $N$ 2$^{\text{nd}}$ order differential equations produces by Lagrange's method. Let us further state that,
\begin{equation}
	\pdv{H}{t} = \dv{H}{t}.
\end{equation}
\section{Euler-Lagrange Equations of the Seecond Kind}
Equations eq. \eqref{eq:euler lagrange} are known as the Euler-Lagrange equations of the first kind. Let us consider a general function $f(y(x), y'(x), x)$ and its total derivative in $x$,
\begin{equation}
	\dv{f}{x} = \pdv{f}{x} + \pdv{f}{y}y' + \pdv{f}{y'}y''.
\end{equation}
If $y$ is a solution of the Euler-Lagrange equations, then eq. \eqref{eq:sols} holds, and,
\begin{equation}
	\begin{split}
	\dv{f}{x} & = \pdv{f}{x} + \dv{x}\left(\pdv{f}{y'}\right)y' + \pdv{f}{y'}y'' \\
	& = \pdv{f}{x} + \dv{x}\left(y'\pdv{f}{y'}\right) \label{eq:fhdso}
	\end{split}
\end{equation}
where the latter terms were reduced by the reverse product rule. By trivial rearranging, eq. \eqref{eq:fhdso} becomes the Euler-Lagrange equations of the second kind,
\begin{equation}
	\boxed{\pdv{f}{x} = \dv{x}\left(f - y'\pdv{f}{y'}\right)}. \label{eq:E-L 2nd}
\end{equation}
Applying eq. \eqref{eq:E-L 2nd} to the Lagrangian,
\begin{equation}
	\begin{split}
	&\dv{t}\underbrace{\left(L - \Dot{x}\pdv{L}{\Dot{x}}\right)}_{-H} = \underbrace{\pdv{L}{t}}_{-\pdv{H}{t}} \\
	\implies & \dv{H}{t} = \pdv{H}{t}.
	\end{split}
\end{equation}
\section{Theory of (Galilean) Relativity}
The postulates of Galilean relativity state that there is no preferred,
\begin{enumerate}
	\item Origin of time;
	\item Position in space;
	\item Inertial frame of reference;
	\item Oritentation in space;
\end{enumerate}
which hold in classical physics. Let us analyse the consequences of these postulates in the Hamiltonian. Let us first define our system:
\\\\
\textit{Consider an isolated system of N particles of mass $m_j$ with vector coordinates $\vb{r}_j$ and momenta $\vb{p}_j$.}
\\\\
For a system of $N$ particles in 3 dimensions, we have $3N$ degrees of freedom and $6N$ independent variables. The total momentum can be defined,
\begin{equation}
	\vb{P} = \sum_j\vb{p}_j.
\end{equation}
We can then summarise the consequences of the postulates,
\begin{enumerate}
	\item[1.] Equations of motion are unchanged by a displacement in time, i.e., $t \to t + \delta t$. This requires the Hamiltonain to be explicitly time independent,
	\begin{equation}
		\pdv{H}{t} = 0 = \dv{H}{t}
	\end{equation}
	implying $H$ is a conserved quantity.
	\item[2.] Equations of motion are unchagned by a displacement of the entire system in space. We must then requiure $H(q_i,p_i)$ to only depend on relative position,
	\begin{equation}
		\vb{r}_j - \vb{r_k} = \vb{r}_{jk} \equiv \vb{r}_j
	\end{equation}
	\item[4.] Equations of motion are invariant under spacial rotations. We require $H$ to only depend on scalar products of $\vb{r}_{jk}$.
\end{enumerate}
We will analyse these in more detail in the sections below.
\subsection{Time Dependence}
Consider a genreal functions, $F(q_i,p_i,t)$. The general equation for change of the function with time is,
\begin{equation}
	\begin{split}
	\dv{F}{t} & = \sum_i\left(\pdv{F}{q_i}\Dot{q}_i + \pdv{F}{p_i}\Dot{p}_i\right) + \pdv{F}{t} \\
	\text{Hamilton's Equations} \implies & = \underbrace{\sum_i\left(\pdv{F}{q_i}\pdv{H}{p_i}-\pdv{F}{p_i}\pdv{H}{q_i}\right)}_{\left[F,H\right]} + \pdv{F}{t}
	\end{split} \label{eq:dghsalksd}
\end{equation}
Above, we have defined the \textit{possoin bracket}, denoted as $\left[F,H\right]$. Let us note that $\left[F,G\right] = - \left[G,F\right]$ for any function $F$ and $G$. Let us then rewrite eq. \eqref{eq:dghsalksd} fully,
\begin{equation}
	\boxed{\pdv{F}{t} = \left[F,H\right] + \pdv{F}{t}} \label{eq:important}
\end{equation}
$\implies$ \textit{The Hamiltonian generates a displacement in time.}.
\\\\
If $\pdv{F}{t} = 0$, and $\left[F, H\right]$, then we say $F$ commutes and is a \textit{conserved quantity}. Furthermore, the Poisson bracket for the set $\left\{\left\{q_i\right\},\left\{q_i\right\}\right\}$ is given by,
\begin{equation}
	\left[q_{\alpha},p_{\beta}\right] = \delta_{\alpha\beta}.
\end{equation}
\subsection{Spatial Dependence}
Let us denote the generalised coordinates $q_i$ as $q_{sj}$, where $s$ represents the spacial direction, and $j$ denotes the particle. Let us consider a displacement of $N$ particles in the $x$ direction,
\begin{equation}
	q_{xj} \to q_{xj} + \delta x, \forall j \in \mathbb{N}
\end{equation}
Our function $F$ then undergoes a displacement,
\begin{equation}
	F \to F + \delta F
\end{equation}
where we can approximate $\delta F$ by Taylor expansion,
\begin{equation}
	\delta F = \sum_i \pdv{F}{q_{xj}}\delta x.
\end{equation}
Let us assume that this displacement is generated by the total $x$ momentum, $P_x = \sum_k p_{xk}$. Let us study the behaviour of the Poisson bracket,
\begin{equation}
	\begin{split}
		\left[F,P_x\right] & = \sum_j\sum_s\left(\pdv{F}{q_{sj}}\underbrace{\pdv{P_x}{P_{sj}}}_{\delta_{sx}} - \pdv{F}{P_{sj}}\underbrace{\pdv{P_x}{q_{sj}}}_0\right) \\
		& = \sum_j \pdv{F}{q_{xj}} \\
		\implies \delta F & = \left[F,P_x\right]\delta x
	\end{split}
\end{equation}
which in words, states, \textit{$P_x$ generates a displacement in the x-direction of the entire system}.
\\\\
$P_x$ can be applied to any observable, so let us apply it to the Hamiltonian. We have,
\begin{equation}
	\Delta H = \left[H, P_x\right]\delta x
\end{equation}
Spatial displacements leave the Hamiltonian unchanged, so we have,
\begin{equation}
	\left[H, P_x\right] = 0.
\end{equation}
We furthermore have the relation in eq. \eqref{eq:important}, when applied to $P_x$,
\begin{equation}
	\dv{P_x}{t} = \left[P_x, H\right] = -\left[H,P_x\right] = 0
\end{equation}
which reveals linear momentum is a conserved quantity.
\subsection{Spatial Orientation}
Generator of rotations about an axis is the component of angular momentum along that axis. Angular momentum is conserved.
\subsection{Independence of Reference Frame}
A boost is a change of intertial frame by a constant velocity. We wish to find the generator of boosts. Let us call this quantity $B$,
\begin{equation}
	\boxed{B = \sum_i\left(m_ix_i - tp_i\right)}
\end{equation}
which is the position of where the centre of mass was at $t=0$, multiplied by the total mass, $M = \sum_i m_i$. We wish to use this quantity as it is conserved. Let us consdier the possion bracket of an arbitrary function $F$ with $B$,
\begin{equation}
	\begin{split}
		\left[F,B\right] & = \sum_i\left(\pdv{F}{q_i}\pdv{B}{p_i} - \pdv{B}{q_i}\pdv{F}{p_i}\right) \\ 
		& = sum_i\left(\pdv{F}{q_i}\left(-t\right) - \pdv{F}{q_i}\left(m_i\right)\right)
	\end{split}
\end{equation}
Let us now set $F = q_{\alpha}$. A small displacement to this generalised coordinate can be written as, $\delta q_k = -t\varepsilon$, where $\varepsilon$ is a small velocity. The transformation caused by the generator $B$ in the $x$-coordinate is then,
\begin{equation}
	x_{\alpha} \to x_{\alpha} - \varepsilon t.
\end{equation}
Similarly, let us consider $F = p_{\alpha}$. The small displacement in this generalised momentum is, $p_{\alpha} = -m_{\alpha}\varepsilon$. So, the transformation caused by the generator $B$ in the $x$ component of momentum is,
\begin{equation}
	p_{\alpha x} \to p_{\alpha x} - m\varepsilon
\end{equation}
In summary, \textit{$B$ generates boosts in the x direction of velocity $\varepsilon$.}
\subsubsection{Boost generator behaviour with the Hamiltonian}
Given that $B$ is a conservative quantity, we must write,
\begin{equation}
	\dv{B}{t} = \left[H,B\right] + \pdv{B}{t} = 0.
\end{equation}
Rearranging this,
\begin{equation}
	\left[H,B\right] = -\pdv{B}{t} = \sum_ip_i = P. \label{eq:HB}
\end{equation}
In order for eq. \eqref{eq:HB} to hold, we require $H \propto P^2$, i.e., kinetic energy depending quadratically on momenta is a consequence of the principle of relativity.
\subsection{Other Symmetries of the Lagrangian and the Hamiltonian}
\begin{enumerate}
	\item $L$ is invariant under the adding of a constant independent of $q_i, \Dot{q}_i, t$.
	\item $L$ is invariant under scaling, i.e., $L$ is invariant under a transformation $L' = KL$ where $K \neq K(q_i, \Dot{q}_i, t$.
	\item Invariant under the addition of a term of the form $\dv{f}{t}$ where $f = f(q_i, t)$. I.e., invariant under transformation $L' = L + \dv{f}{t}$.
	\begin{proof}
		Consider the action,
		\begin{equation}
			\begin{split}
				S' = \int_{t_1}^{t_2}L'(q_i, \dot{q}_i, t)\dd{t} & = \int_{t_1}^{t_2}L(q_i, \Dot{q}_i, t)\dd{t} + \int_{t_1}^{t_2}\dv{f}{t} \dd{t} \\
				& = S + \left[f\right]_{t_1}^{t_2} \\
				& = S + f(q_i(t_1), t_1) - f(q_i(t_2), t_2)
			\end{split}
		\end{equation}
		and the latter terms only depend on the end points of the action. These points are fixed, which imply that the function $f$ does not depend on the path. So we have that $\delta S' = \delta S = 0$, and the action after the transformation corresponds to the same path as before the transformation. So the equations of motion remain unchanged.
	\end{proof} 
\end{enumerate}
\chapter{Normal Modes for Lagrangian Dynamics}
Normal modes are the modes of the system where all parts of the system are oscillating at the same frequency. "Normal" refers to orthogonality of the modes, meaning that there is no coupling between the modes. If a system oscillates in one of its normal modes, it will stay in that mode.
\begin{figure}
	\centering
	\begin{tikzpicture}[]
		\draw (-0.5,2) -- (2.5,2);
		\draw (0,0) to[cute inductor,
		inductors/coils=9, inductors/width=1.4,
		nodes width=0.1, l=$s$, 
		o-o] (2,0);
		\draw (0,0) -- ++(0,2) node[left, midway]{$l$};
		\draw (2,0) -- ++(0,2) node[right, midway]{$l$};
		\draw (0,-.5) node[flowarrow, label=below:$x_1$]{};
		\draw (2,-.5) node[flowarrow, label=below:$x_2$]{};
	\end{tikzpicture}
	\caption{}
	\label{fig:coupled pendula}
\end{figure}
\\\\
We will consider 3 different methods of finding normal modes by analysing a simple coupled pendulum. Let us first state the lagrangian for this system,
\begin{equation}
	L = \frac{1}{2}m\left(\Dot{x}^2_1 + \Dot{x}_2^2\right) - \frac{1}{2}k(x_2 - x_1)^2 - \frac{1}{2}\frac{mg}{l}
\end{equation}
with the corresponding equations of motion,
\begin{align}
	m\Ddot{x}_1 & = -k(x_1 - x_2) - \frac{mg}{l}x_1 \label{eq:1} \\
	m\Ddot{x}_2 & = +k(x_1 - x_2) - \frac{mg}{l}x_2 \label{eq:2}
\end{align}
\section{Informal Method}
Let us add eq. \eqref{eq:1} and eq. \eqref{eq:2}, and define $X_1 = x_1 + x_2$,
\begin{equation}
	m\Ddot{X_1} = -m\frac{g}{l}X_1
\end{equation}
from which we can clearly identify SHM in $X_1$ with $\omega^2 = \frac{g}{l}$, corresponding to the first normal mode.
\\\\
Let us now subtract the two equations of motion and define $X_2 = x_1 - x_2$,
\begin{equation}
	\Ddot{X}_2 = -\left(\frac{2k}{m} + \frac{g}{l}\right)X_2
\end{equation}
which is clearly SHM in $X_2$ with $\omega^2 = \frac{2k}{m} + \frac{g}{l}$.
\section{More Formal Method}
In normal modes, all components oscillate with the same frequency. We can then assume a general solution,
\begin{equation}
	x_j = C_j e^{i\omega_n t} \implies \Ddot{x}_j = -\omega_n^2C-j e^{i\omega_n t}.
\end{equation}
Doing this transforms a simultaneous differential equation into a purely algebraic form. Substituting this into our equations of motion and dividing through by $e^{i\omega_nt}$, 
\begin{align}
	-m\omega_n^2C_1 & = - k(C_1 -C_2) - m\frac{g}{l}C_1 \\
	-m\omega_n^2C_2 & = + k(C_1 - C_2) - m\frac{g}{l}C_2.
\end{align}
We then obtain two solutions,
\begin{align}
	C_2 = C_1 && C_2 = -C_1
\end{align}
which correspond to the pendula moving in the same direction with equal amplitude, and moving in opposite directions with equal amplitude. Our solutions for the normal modes are as in the previous section.
\section{Formal Matrix Treatment}
Let us consider the equations of motion again, assuming that all parts of the system are performing simple harmonic motion,
\begin{align}
	m\Ddot{x}_1 = -m\omega_n^2x_1 = -\left(\frac{mg}{l} + k\right)x_1 -(-k)x_2\label{eq:12} \\
	m\Ddot{x}_1 = -m\omega_n^2x_2 = -(-k)x_1 -\left(\frac{mg}{l} + k\right)x_2 \label{eq:22} \\
\end{align}
We can clearly rewrite these equations as a single matrix equation,
\begin{equation}
	M\Ddot{X} = -\omega_n^2MX = -KX
\end{equation}
if we define,
\begin{align}
	X = \begin{pmatrix}
		x_1 \\ x_2
	\end{pmatrix} && M = \begin{pmatrix}
	m & 0 \\ 0 & m
	\end{pmatrix} && K = \begin{pmatrix}
	\frac{mg}{l}+k & -k \\
	-k & \frac{mg}{k}+k
	\end{pmatrix}
\end{align}
where $X$ is the vector of our degrees of freedom, $M$ is our mass matrix, and $K$ encodes the restoring forces acting on the system. We are further able to identify the equation,
\begin{equation}
	KX = \omega_n^2MX. \label{eq:293}
\end{equation}
Mutliplying eq. \eqref{eq:293} by $M^{-1}$,
\begin{equation}
	\boxed{M^{-1}K X = \omega_n ^2X} \label{eq:matrix eq motion}
\end{equation}
which is the general for motion \textit{close to an equilibrium} in matrix form, which is an eigen equation. The eigenvalues of this equation represent the normal frequencies, i.e., solving,
\begin{equation}
	\det{M^{-1}K - \omega_n^2I} = 0.
\end{equation} 
Evaluating this produces the normal frequencies as in the previous sections. By further finding the eigenvectors of the eigenequation, we are able to identify the normal modes of the system. For our particular system with the coupled pendula, we find that the eigenvectors are,
\begin{align}
	X_1 = \frac{1}{\sqrt{2}}\begin{pmatrix}
		1 \\ 1
	\end{pmatrix}C && X_2 = \frac{1}{\sqrt{2}}\begin{pmatrix}
	1 \\ -1
	\end{pmatrix}C
\end{align}
where $C$ is the amplitude of the oscillation. We can interpret the first normal mode corresponds to the two masses moving in the same direction, while the second normal mode corresponds to the two masses moving in opposite directions with equal amplitude.
\\\\
In order to write the general displacement of a mass $m_j$ in the $n$th normal mode, we must take into account the amplitude and phase,
\begin{equation}
	x_j^{(n)} = X_n^{(j)}C_ne^{i(\omega_n t)}
\end{equation}
and to get the total displacement of the mass $m_j$, we must sum over all normal modes,
\begin{equation}
	x_j = \sum_n\left(Q_n^{(j)} C_ne^{i(\omega_nt)}\right)
\end{equation}
where we determine $C_n$ from intial conditions. $C_n$ can be complex in order to encode the phase differencec. NOTE: We require 2N conditions for $N$ degrees of freedom.
\subsection{Lagrangian in Matrix Form}
The Lagrangian in matrix form is given by,
\begin{equation}
	\boxed{L = \frac{1}{2}\Dot{X}^TM\Dot{X} - \frac{1}{2}X^TKX}
\end{equation}
\subsection{Diagonalisation of the equations of motion}
We can diagonalise our equations of motion. If eigenvectors of our equations of motion are orthogonal, such that,
\begin{equation}
	\vb{X}^{(j)}\cdot\vb{X}^{(k)} = \delta_{ij}
\end{equation}
then we can construct an orthogonal matrix, whose columns are the eigenvectors of the system,
\begin{equation}
	B_{ij} = \vb{X}_i^{(j)}.
\end{equation}
From this we can diagonalise our equation of motion, i.e., eq. \eqref{eq:matrix eq motion},
\begin{equation}
	\boxed{\underbrace{(B^{-1}M^{-1}KB)}_{\substack{\text{Diagonal}\\\text{Matrx}}}\underbrace{(B^{-1}\vb{X})}_{Q} = \omega_n^2 \underbrace{\left(B^{-1}X\right)}_{Q}}
\end{equation}
which produces equations of motion whcih are diagonal in the normal modes of our system, or viewing this another way, in the normal coordinates of the system. We can further diagonalise the Lagrangian,
\begin{equation}
	\boxed{L = \frac{1}{2}\Dot{Q}^TB^TMB\Dot{Q} - \frac{1}{2}\Dot{Q}^TB^TKBQ}
\end{equation}
This allows for the normal modes of the system to be decoupled.
\subsection{Zero-Point Modes}
Zero point modes occur for eigenvalues of $\lambda = 0$. In a system close to equilibrium, this means $\omega^2 = 0$. We have,
\begin{equation}
	\Ddot{\vb{X}} = \omega^2\vb{X} \implies \Ddot{\vb{X}} = 0.
\end{equation}
The normal mode is then,
\begin{equation}
	\Dot{x_1} = \Dot{x_2} = \Dot{x_3} = \text{Const.}.
\end{equation}
which is known as free motion.
\subsection{Degenderate Normal Modes}
Degenerate modes occur when we have a solution to the eigenvector equation such that,
\begin{equation}
	x_1 + x_2 + x_3 = 0 \label{eq:cond}
\end{equation}
which implies that there are infinitely many eigenvectors, so long that they satisfy eq. \eqref{eq:cond}. Furthermore, the linear of two degenerate normal modes always produces another degenerate normal mode which satisfied eq. \eqref{eq:cond}.
\appendix
\chapter{Misc. Forumlae}
\section{Total Derivative}
\begin{equation}
	\dv{f}{t} = \sum_i \pdv{f}{r_i}\dv{r_i}{t} \label{eq:totalderivative}
\end{equation}
for $r_i = r_i(t)$.
\end{document}
